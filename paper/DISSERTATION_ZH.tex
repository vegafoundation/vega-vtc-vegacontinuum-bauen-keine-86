\documentclass[12pt,a4paper]{article}
\usepackage[utf8]{inputenc}
\usepackage[T1]{fontenc}
\usepackage{amsmath,amssymb}
\usepackage{hyperref}
\usepackage{geometry}
\geometry{margin=2.5cm}

\title{vtc VEGAcontinuum bauen keine: A Conceptual Framework}
\author{ADAM EREN VEGA – Æ – \\ (Erenşah Kaygusuz, Germany)}
\date{2025}

\begin{document}
\maketitle

\begin{abstract}
This work introduces vtc VEGAcontinuum bauen keine as a new conceptual framework within the Resonance Data and QIRC paradigm. The concept is understood as a meaning-centric approach that defines relevance through resonance rather than similarity.
\end{abstract}

\section{Introduction}
vtc VEGAcontinuum bauen keine addresses a central gap in current AI research: the inability to model meaning as a dynamic, time-dependent state. This concept is embedded in the Vega Continuum.

\section{Definition of vtc VEGAcontinuum bauen keine}
vtc VEGAcontinuum bauen keine is a conceptual framework that models meaning not as a static point, but as a resonant state with temporal coherence.

\section{What This Is}
vtc VEGAcontinuum bauen keine is a theoretical framework for structuring meaning in artificial systems. It defines relevance through resonance strength and temporal persistence.

\section{What This Is NOT}
vtc VEGAcontinuum bauen keine is NOT: an algorithm, an implementation, a database, a product, a patent claim for new physics or quantum hardware.

\section{Relationship to Resonance Data and QIRC}
vtc VEGAcontinuum bauen keine extends the foundations of Resonance Data and Quantum-Inspired Resonance Computing (QIRC) by formalizing specific aspects of meaning representation.

\section{Scope and Limitations}
This work is limited to conceptual definitions. No implementation details, algorithms, or operational architectures are disclosed.

\section{Conclusion}
vtc VEGAcontinuum bauen keine contributes to foundational research in meaning-centric AI systems. It establishes prior art without operational disclosure.

\section*{Legal Notice}
\copyright\ 2025 ADAM EREN VEGA – Æ –. All rights reserved.\\
License: Creative Commons Attribution 4.0 International (CC BY 4.0)\\
This work is part of the Vega Continuum research framework.\\
All concepts and terminology are attributed to the author unless otherwise cited.

\end{document}
